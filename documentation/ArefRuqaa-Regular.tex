\input bidi
\nopagenumbers
\font\aref="[../\jobname.ttf]:+ss03" at 185pt \aref
\vbox{
\vskip-.8in
  \parindent=-.6in
  الخط
  \vskip.1em
  العربي
  \vskip.1em
  فن
  \vskip-.3em
  \indent\hboxR{\raise.2em\hbox{و}\kern-0.3em تصميم}
}
\vfill
\eject

\tolerance 9999 %
\emergencystretch 3em %
\setRTL
\font\aref="[../\jobname.ttf]" at 30pt \aref
\leftskip 0pt plus 2em\relax
\openup.55em

\noindent
الخط العربي هو فن وتصميم الكتابة في مختلف اللغات التي تستعمل الحروف العربية. تتميز الكتابة العربية بكونها متصلة مما يجعلها قابلة لاكتساب أشكال هندسية مختلفة من خلال المد والرجع والاستدارة والتزوية والتشابك والتداخل والتركيب.

\noindent
ويقترن فن الخط بالزخرفة العربية حيث يستعمل لتزيين المساجد والقصور، كما أنه يستعمل في تجميل المخطوطات والكتب وخاصة لنسخ القرآن الكريم. وقد شهد هذا المجال إقبالاً من الفنانين المسلمين بسبب نهي الشريعة عن رسم البشر والحيوان خاصة في ما يتصل بالأماكن المقدسة والمصاحف.
\bye
